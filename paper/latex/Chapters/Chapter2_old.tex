\chapter{Literature Review}

\label{LiteratureReview}

\section{Machine Learning in Sports Analysis}

Over the past two decades, technology has enabled significant growth in the amount of data collected during sports matches. It is no surprise that machine learning algorithms have become increasingly popular for analysis of these increasingly large and complex datasets. 

Statistical prediction of sports outcomes, such as the match winner or final score, is of great interest to spectators, teams, and other industry stakeholders. In 2020, Horvat and Job analysed 38 research papers aimed at this particular application for various team sports, ranging from basketball, football, cricket, baseball, and American football. They found that feature selection was critical for building effective models; earlier studies relied solely on expert opinions, with newer studies using statistical techniques (such as regularization) for feature selection. 

The machine learning models employed by researchers differ depending on the nature of the data and the complexity of the problem: the various forms of Neural Networks are popular for predictions, as they can effectively model non-linear relationships by capturing complex patterns in the data. Other studies used Support Vector Machines, Decision Trees \& Random Forests, Logistic Regression, and K-Nearest Neighbours to varying effect. Models were evaluated either by a chronological segmentation of the dataset, or by using k-fold cross validation. The success of the model is heavily dependent on the sport, the nature of the data, and the context of the problem. \cite{mlsports}

Do the same findings apply to esports analysis? Jenny et al. compared esports to traditional sports and argued that, other than the degree of physicality involved, esports display the same characteristics of traditional sports: esports are a voluntary and intrinsically motivated activity, governed by rules, in which opposing players compete. Furthermore, they require skills such as strategic thinking, hand-eye coordination, and rapid decision-making. They concede that the degree of physicality differs significantly, as esports involve fine motor skills as opposed to the high levels of physical exertion typically required in traditional sports. Their research also acknowledged that esports have a global following and are becoming increasingly institutionalized, further cementing their legitimacy \cite{def-esports}. It is therefore reasonable to assume that the techniques described in \cite{mlsports} should, to some extent, be applicable to esports as well. 

\section{Machine Learning in Esports Analysis}

In the age of \textit{big data}, esports have a unique advantage over traditional sports - they are natively digital. This means that recording data from a given match is much cheaper and easier, and even amateur games can be recorded in fine detail. It is therefore unsurprising that esports provide fertile grounds for data analysis. In the literature there is also a particular emphasis on real-time match analysis.



\cite{trueskill-csdota}

% demonstrate understanding of literature around research question
% show that there is a gap in the research, how can you take a unique angle, establish clear need for research
% inform the methodology